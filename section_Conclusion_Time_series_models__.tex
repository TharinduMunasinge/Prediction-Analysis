\section{Conclusion}

To provide a better QoS while maintain lower resource leasing cost in PaaS system is challenging in reactive, threshold based autoscaling solutions. Even Though  proactive solutions overcome these challenges, it require better prediction mechanism. Existing researches on  workload prediction techniques mainly focus on single model offline-trained. For heterogeneous  applications with drastically different  and continuously growing workload time series, forecasting method that includes a single model, would not provide sufficiently accurate results while generate larger errors on some workload patterns .When there is such uncertainty in finding the best model , combining may reduce the instability of the forecast and therefore improve prediction accuracy. In this work we proposed an ensemble prediction technique that would combine predicted values from different prediction techniques based on the past inverse forecasting error in each individual models. We train each individual model real time (as in autoscaler) and combine the forecasts by the weights calculated from the inverse error of fitted values for trained data. Without taking only forecasting last error or mean over the all past errors , we used exponential smoothing to fit the past forecasting error of each model.We observed that using the smoothed error as the contribution from each model, significantly improves the accuracy.  We implemented proposed ensemble technique using three popular forecasting models : Neural Network, ARIMA and Exponential Model. Tested on three publicly available datasets, two cloud workload dataset and Google cluster dataset.  We saw that our approach made the least errors in couple datasets while never performing worse than the individual prediction methods. Since proposed forecasting method should generate the forecast within bounded time period, we have limit the size of input training window, but the optimum size of the input window that would improve the accuracy while maintaining temporal restriction on calculating forecast real-time,  need to be researched as an improvement for this technique.We should also point out that it is not a good idea to blindly combine all possible models available.

