\section{Related Work}

There is a significant number of already established research work in the workload prediction domain. Exponential smoothing is popularly used for prediction. Mi et al. \cite{Mi_2010} also used quadratic exponential smoothing against real workload traces such as World Cup 98 \cite{WorldCup_1998}, and showed good results while minimizing errors. The auto-regression method has also been widely used in the literature \cite{Kupferman_2009}, \cite{Khatua_2010}, \cite{Zhenhuan_Gong_2010}. Kupferman et al. \cite{Kupferman_2009} applied auto-regression of single order to predict the request rate and found that its performance depends largely on several user-defined parameters such as the size of the input window and the horizon window. Auto-Regressive Moving Average (ARMA) method is the dominant time series analysis technique for workload and resource usage prediction. Roy et al. \cite{Roy_2011} used a second order ARMA model for workload prediction on the World Cup 98 traces and showed accurate results. While machine learning techniques like neural networks and regression have been applied by several authors, these techniques are not used as widely as time series methods. Several work also focus on using history window values as the input for a neural network \cite{Islam_2012} and a multiple linear regression equation \cite{Kupferman_2009}, \cite{Islam_2012}. However, the accuracy of both methods depends on the input window size.

Time-series forecasting can also be combined with reactive techniques. For example, Iqbal et al. \cite{Iqbal_2011} proposed a hybrid scaling technique that utilizes reactive rules for scaling up (based on CPU usage) and a regression-based approach for scaling down.

Another important parameter in workload prediction is estimating the prediction interval/window, $r$. Islam et al. \cite{Islam_2012} proposed using a 12-minute interval, because the set-up time of VM instances in the cloud was typically around 5-15 minutes at that time.

The related work outlined above primarily focus on a specific dataset. Hence, the resulting predictive model might not be a good fit for scenarios involving multiple applications with drastically different workload characteristics. In some of the solutions the predictive model has been derived using offline training on datasets, so that dynamic adaptation to current workload changes may not be possible. However, these may not work well on PasS platforms as the same platform may be used to build different application with drastically different workload characteristics.

From a mathematical perspective, certain preconditions have to be fulfilled by a given dataset for it to be fitted satisfactorily using time series prediction techniques like exponential moving average and ARIMA. It is quite unlikely that datasets from all applications will meet these preconditions. Therefore, a given prediction technique may require its parameters to be adjusted to fit to specific datasets. Hence, identifying optimum parameters for the model should happen dynamically via an online training process, rather than using predefined constants defined during the offline training.