\section{Related Work}

There is a significant amount of already established research work in the workload prediction domain. Exponential smoothing is popularly used for predicting purposes. Mi et al. [18] used quadratic exponential smoothing against real workload traces (World Cup 98), and showed accurate results, with a small amount of error. The auto-regression method has also been widely used in the literature ([19], [20], [21]). Kupferman et al. [19] applied auto-regression of single order to predict request rate and found that its performance depends largely on several user-defined parameters: the size of the input window and the size of the horizon window. Auto-regressive moving average method (ARMA) is the dominant time series analysis technique for workload and resource usage prediction. Roy et al. [4] used second order ARMA for workload prediction on the World Cup 98 traces and showed accurate results. Machine learning techniques like neural networks and regression have been applied by several authors but those techniques are not used as widely as time series methods. Researches have also been conducted on using history window values as the input for a neural network ([22]) and a multiple linear regression equation ([19], [22]). The accuracy of both methods depends on the input window size.

Another interesting point in workload prediction is estimating the prediction interval, $r$. Islam et al. [22] proposed using a 12-minute interval, because the set-up time of VM instances in the cloud was typically around 5-15 minutes by that time.

Time-series forecasting can be combined with reactive techniques. Iqbal et al. [23] proposed a hybrid scaling technique that utilizes reactive rules for scaling up (based on CPU usage) and a regression-based approach for scaling down.

\subsection{Limitations of existing research work}

The researches considered so far are primarily based on specific datasets. Hence the resulting predictive model might not be a good fit for an application with different workload characteristics. In some of the solutions the predictive model has been found using offline training on datasets so that dynamic adaptation for current workload changes is not possible. 
From a mathematical perspective, to be fitted accurately using some time series prediction techniques like exponential moving average and ARIMA, the dataset needs to fulfill certain preconditions. It is quite unlikely that all datasets generated by different applications will meet these preconditions. Even a given prediction technique may incorporate parameters which need to be adjusted in order to accurately fit a specific data set. Hence identifying optimal parameters for the model should happen dynamically as an online training process, rather than having predefined constants from an offline training process.