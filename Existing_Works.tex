\section{Related Work}

There is a significant amount of already established research work in the workload prediction domain. Exponential smoothing is popularly used for predicting purposes. Mi et al. \cite{Mi_2010} also used a quadratic exponential smoothing against real workload traces (World Cup 98 \cite{worldcup}), and showed good results with a small amount of error. The auto-regression method has also been widely used in the literature (\cite{key}, \cite{Khatua_2010}, \cite{Zhenhuan_Gong_2010}). Kupferman et al. \cite{key} applied auto-regression of single order to predict request rate and found that its performance depends largely on several user-defined parameters: the size of the input window and the size of the horizon window. Auto-regressive moving average method (ARMA) is the dominant time series analysis technique for workload and resource usage prediction. Roy et al. \cite{Roy_2011} used a second order ARMA for workload prediction on the World Cup 98 traces and showed accurate results. Machine learning techniques like neural networks and regression have been applied by several authors but those techniques are not used as widely as time series methods. Some researches also focus on using history window values as the input for a neural network \cite{Islam_2012} and a multiple linear regression equation (\cite{key}, \cite{Islam_2012}). The accuracy of both methods depends on the input window size.

Another interesting point in workload prediction is estimating the prediction interval, $r$. Islam et al. \cite{Islam_2012} proposed using a 12-minute interval, because the set-up time of VM instances in the cloud was typically around 5-15 minutes by that time.

Time-series forecasting can be combined with reactive techniques. Iqbal et al. \cite{Iqbal_2011} proposed a hybrid scaling technique that utilizes reactive rules for scaling up (based on CPU usage) and a regression-based approach for scaling down.

\subsection{Limitations of existing research}

The researches outlined above primarily focus on specific datasets. Hence the resulting predictive model might not be a good fit for scenarios involving multiple applications with different workload characteristics. In some of the solutions the predictive model has been derived using offline training on datasets, so that dynamic adaptation for current workload changes may not be possible.

From a mathematical perspective, certain preconditions have to be fulfilled by a given dataset for it to be fitted satisfactorily using time series prediction techniques like exponential moving average and ARIMA. It is quite unlikely that datasets from all applications will met these preconditions. Even a given prediction technique may require its parameters to be adjusted in order to be fitted with specific datasets. Hence identifying optimum parameters for the model should happen dynamically via an online training process, rather than using predefined constants defined during offline training.