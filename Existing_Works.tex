\section{Existing Works}

There is a significant amount of already established research work in the workload prediction domain. Exponential smoothing is popularly used for predicting purposes. Mi et al. [18] also used a quadratic exponential smoothing against real workload traces (World Cup 98), and showed good accurate results, with a small amount of error. The auto-regression method has been widely used in the literature ([19], [20], [21]). Kupferman et al. [19] applied autoregression of single order to predict request rate and found that its performance depends largely on several user-defined parameters: the size of the input window and the size of the horizon window.Auto-regressive moving average method (ARMA) is the dominant time series analysis technique for workload and resource usage prediction. Roy et al. [4] use a second order ARMA for workload prediction on the World Cup 98 traces and showed good accurate results. Machine learning techniques like neural networks and regression have been applied by several authors but those techniques are not widely used as time series methods. There were researches on using history window values as the input for a neural network [22] and a multiple linear regression equation [19], [22]. The accuracy of both methods depends on the input window size.

\subsection{Prediction horizon}

Prediction Horizon
Another interesting point in workload prediction is estimating the prediction interval, r. Islam et al. [22] proposed using a 12-minute interval, because the set-up time of VM instances in the cloud was typically around 5-15 minutes by that time. Time-series forecasting can be combined with reactive techniques. Iqbal et al. [23] proposed a hybrid scaling technique that utilizes reactive rules for scaling up (based on CPU usage) and a regression-based approach for scaling down.

\subsection{Limitations of the existing research works}

 These research works were primarily based on specific datasets. So that the resulted predictive model might not be a good fit for an application with different workload characteristics. In some of the solutions the predictive model was found using offline training on datasets so that the dynamic adaptation for the current workload changes is not there. 
In mathematically some time series prediction techniques like exponential moving average and ARIMA has some preconditions to be fitted well with the dataset. It’s quite unlikely that all the applications will met these preconditions. Even in a single prediction technique, there are parameters to be adjusted to fit with specific data set. So identifying optimum parameters for the model should be happend dynamic online training process rather than having constants defined in offline training process.
