\section{Related Work}

There is a significant number of already established research work in the workload prediction domain.  Kupferman et al. \cite{Kupferman_2009} applied single order auto-regression to predict the request rate and found that its accuracy depends on several parameters such as the size of the input window and the horizon window. Exponential smoothing is popularly used for prediction. Mi et al. \cite{Mi_2010} also used quadratic exponential smoothing against real workload traces such as World Cup 98 \cite{WorldCup_1998}, and showed good results. Auto-Regressive Moving Average (ARMA) method is one of the dominant time series analysis techniques for workload and resource usage prediction. Roy et al. \cite{Roy_2011} used a second order ARMA filter for workload prediction on the World Cup 98 traces and showed accurate results.  

Machine learning techniques like neural networks, regression and Hidden Markov Model have been applied by several authors for workload prediction. Yang et al.\cite{Yang_2013} have used use a sliding window based linear regression model (LRM) for workload prediction and showed a lower prediction deviation  . Khan et al. \cite{Khan_2012} used Hidden Markov Model (HMM) to explore the temporal correlations in workload pattern changes . Several work also focus on using history window values as the input for a neural network \cite{Islam_2012} However, the accuracy of such method depends on the input window size.



\textbf{Limitations :} The related work outlined above primarily focus on a specific dataset. Hence, the resulting predictive model might not be a good fit for scenarios involving multiple applications with drastically different workload characteristics. In some of the solutions the predictive model has been derived using offline training on datasets, so that dynamic adaptation to current workload changes may not be possible. However, these may not work well on PasS platforms as the same platform may be used to build different application with drastically different workload characteristics.

From a mathematical perspective, certain preconditions have to be fulfilled by a given dataset for it to be fitted satisfactorily using time series prediction techniques like exponential moving average and ARIMA. It is quite unlikely that datasets from all applications will meet these preconditions. Therefore, a given prediction technique may require its parameters to be adjusted to fit to specific datasets. Hence, identifying optimum parameters for the model should happen dynamically via an online training process, rather than using predefined constants defined during the offline training.