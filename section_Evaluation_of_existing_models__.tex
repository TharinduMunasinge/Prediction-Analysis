\section{Evaluation of existing models}

In this analysis we wanted to qualitatively analyze the ability of individual prediction technique to learn from current workload history and predict the time series while time series data is evolving periodically as what happen in autonomous auto-scaler in operation. We simulated an online training scenario by streaming data points of the time series one at a time to each method and plotted the one-step lookahead prediction from the resulting models at each stage, against the real data points. 

\subsection{datasets}
Since the objective of this experiment to identify how well an individual method can predict the future values by only considering the past history. we tried several publicly available datasets in the forecast package with predictable patterns even though they are not really related to the cloud workloads. Neverthless the experimental results elaborated in Section 5 in which we  compare and contrast the performance of the proposed method vs existing methods. , are based on two real-world cloud workloads, Google cluster data  in addition to the datasets in forecast package. For evaluation purpose we used forecast package in R \cite{forecastPackage} in which there are time series datasets in different domains with typical patterns like trends, cycles and seasonality factors. 

\begin{itemize}
\item airmiles : The revenue passenger miles flown by commercial airlines in the United States for each year from 1937 to 1960.
\item euretial: Quarterly retail trade index in the Euro area (17 countries), 1996-2011, covering wholesale and retail trade, and repair of motor vehicles and motorcycles. (Index: 2005 = 100).
\item sunspotarea: Annual averages of the daily sunspot areas (in units of millionths of a hemisphere) for the full sun. Sunspots are magnetic regions that appear as dark spots on the surface of the sun. The Royal Greenwich Observatory compiled daily sunspot observations from May 1874 to 1976. Later data are from the US Air Force and the US National Oceanic and Atmospheric Administration.
\item oil: Annual oil production (millions of tonnes), Saudi Arabia, 1965-2010.
\end{itemize}



\subsection{implementation of methods}
For this evaluation we have used the implementations of the Forecast pacakge \cite{forecastPackage}
ARIMA: According to <<ROBIN>> auto.arima()  find out the best fitted ARIMA model and approximated the model coeffficents in by minimizing the error<<<BESPECIFIC >>> [https://www.otexts.org/fpp/8/7]

Ets: Model the data with best fitted exponential model [https://www.otexts.org/fpp/7/7]
    \cite{Wagner_2011}
	Nnetar: [https://www.otexts.org/fpp/9/3]
