\section{Evaluation of existing models}

In this analysis we wanted to qualitatively analyze the ability of individual prediction technique to learn from current workload history and predict the time series while time series data is evolving periodically as what happen in autonomous auto-scaler in operation. We simulated an online training scenario by streaming data points of the time series one at a time to each method and plotted the one-step lookahead prediction from the resulting models at each stage, against the real data points. 
We chosen three widely used prediction methods in the literature: ARIMA, Neural network, Exponential Models and the existing workload prediction technique in Stratos.

\subsection{Time series Forecasting techniques}



Since the objective of this experiment to identify how well an individual method can predict the future values by only considering the past history. we tried several publicly available datasets in the forecast package with predictable patterns even though they are not really related to the cloud workloads. Neverthless the experimental results elaborated in Section 5 in which we  compare and contrast the performance of the proposed method vs existing methods. , are based on two real-world cloud workloads, Google cluster data  in addition to the datasets in forecast package. For evaluation purpose we used forecast package in R \cite{forecastPackage} in which there are time series datasets in different domains with typical patterns like trends, cycles and seasonality factors. 

\subsection{datasets}



\subsection{implementation of methods}
For this evaluation we have used the implementations of the Forecast pacakge[ID] . <<DESCRIPTON ABOUT THE IMPLEMENTATION>>>>
ARIMA: According to <<ROBIN>> auto.arima()  find out the best fitted ARIMA model and approximated the model coeffficents in by minimizing the error<<<BESPECIFIC >>> [https://www.otexts.org/fpp/8/7]

Ets: Model the data with best fitted exponential model [https://www.otexts.org/fpp/7/7]

    \cite{Wagner_2011}
	Nnetar: [https://www.otexts.org/fpp/9/3]
