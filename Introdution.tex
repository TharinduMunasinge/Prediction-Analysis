\section{Introdution}

Cloud computing is no longer the next revolution, it has already gain the ground with advantages like scalability, high availability and on-demand resource provisioning. According to the NIST definition \cite{Mell_2011} , cloud computing refers to the delivery of computing resources like networks, servers, storage, application and platform over the network based on the demand.Cloud computing mainly based on the virtualization allowing for pooling and dynamically allocating hardware resources. Elasticity is a major feature in cloud platform. This is often achieved via an auto-scaling process, which dynamically allocates and deallocates resources for a cloud application to optimize its resource utilization while minimizing the cost as well as achieving the desired Quality of Service (QoS) and availability goals\cite{Roy_2011}\cite{Armbrust_2010}.

Autoscaling can be achieved  reactive based approach or proactive based approach. In reactive, threshold based auto-scaling, users have to specify the policies that would evaluate when to trigger the scale-up or scale-down events, and scaling would occur only after the scaling event is triggered \cite{Lorido_Botran_2014} .
Although policy-aware auto-scaling seems to be the most customizable approach from the user’s perspective [7], it has several weaknesses \cite{Alipour:2014:AAI:2735522.2735532} :

\begin{itemize}
\item Necessity of prior user expertise (e.g., of traffic patterns, service health patterns and IaaS/PaaS parameters like pricing models and packages) for composing effective rules that can carry out scaling efficiently
\item Existing systems do not consider the startup time of an instance (fault tolerance time) when taking an auto-scaling decision
\item Inability to adapt to workload patterns (e.g., thrashing during load fluctuations)
\end{itemize}


As described in the <<FIRST>> chapter, to avoid the drawbacks of threshold based reactive systems a proactive scaling algorithm should be run on top of the predicted future workload values. our proactive scaling solution has 3 major components

\begin{itemize}
\item Workload Predictor - Predicting the workload for future time horizon
\end{itemize}

\begin{itemize}
\item Scaling decision maker- Calculating the number of virtual machines required to handle the predicted workload while minimizing the cost of the scaling action.
\end{itemize}

\begin{itemize}
\item Cost optimizer - Performing further cost optimizations like smart killing
\end{itemize}

\usepackage{multirow}


\subsection{Challenges}

We have identified several challenges specific to workload prediction in a PaaS platforms.
\begin{itemize}
\item PaaS system can have different applications with different workload characteristics, so that the workload predictor should not be overfitted to a specific workload pattern.
\end{itemize}
\begin{itemize}
\item Workload predictive algorithm should be evolved and continuously learn the workload characteristics . 
\end{itemize}
\begin{itemize}
\item Workload predictor should be able to give results within a bounded time period.
\end{itemize}
\begin{itemize}
\item Give that the  time horizon for the prediction should be chosen based on the constraints like VM uptime and gracefully shutting down time, predictor should be able to give sufficiently accurate results.
\end{itemize}


