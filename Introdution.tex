\section{Introduction}
Cloud computing has already gained ground with advantages like scalability, on-demand resource provisioning, and high availability. According to the NIST definition \cite{Mell_2011}, cloud computing refers to the delivery of computing resources, such as networks, servers, storage, applications, and platforms, over the network based on user demand. Based on this definition, it is evident that the elasticity is a major requirement of any cloud platform, which is often achieved via an auto scaling process. Auto scaling refers to the dynamic allocation and release of resources for a cloud application, in order to optimize its resource utilization while minimizing the cost, as well as achieving the desired Quality of Service (QoS) and availability goals \cite{Roy_2011} \cite{Armbrust_2010}.

It is the responsibility of the application developer to deal with auto scaling if it is operating on the IaaS level. By contrast, if the application itself runs at the PaaS layer, then the burden of auto scaling is removed from the developer and placed onto the PaaS vendor. Operating at the PaaS layer enables the auto scaling services to use high level, application-specific metrics such as request in flight count as well as low level, cloud-specific metrics such as memory consumption. PaaS level auto scaling mainly deals with infrastructure resource management as several applications share and compete for resources simultaneously. Hence it is difficult to auto scale at the PaaS level due to different applications with drastically different workload patterns.  

Auto scaling can be achieved via reactive as well as proactive approaches. In reactive, threshold-based auto scaling, users have to specify thresholds for workload metrics, and scaling would occur only after such thresholds are exceeded \cite{Lorido_Botran_2014}. In this approach, even though it is desirable to maintain high thresholds which would result in higher levels of resource utilization, thresholds should be sufficiently small to compensate for delays in formulating and executing scaling actions. Although this seems to be the most customizable approach from the user’s perspective, it has several weaknesses \cite{Alipour_2014} such as the inability to adapt to workload patterns (e.g., thrashing during load fluctuations), low resource utilization and higher cost under smaller thresholds, and the risk of service/QoS degradation under larger thresholds and rapidly increasing workloads. Coming up with an optimum threshold in a reactive auto-scaling system is nontrivial and it requires experience and expertise (e.g., awareness of traffic patterns and IaaS/PaaS parameters like pricing models and packages) for composing effective thresholds that can carry out scaling efficiently.

In the proactive approach, resource requirements for the future time horizon are forecasted based on the demand history. With accurate predictions, an application can take early scaling decisions so that when the workload reaches a particular level, the required resources would have already been allocated. Consequently, resource utilization can be safely increased to a maximum level. However, the reliability of such an approach depends mainly on the accuracy of the predicted values. Because we forecast the future workload requirement (e.g., memory consumption, CPU utilization, network request count) from historical data of the respective measures, time series forecasting methods are applicable for this requirement.

We have identified following challenges specific to workload prediction for auto scaling:
\begin{itemize}
\item A PaaS cloud system may be used to build different applications with vastly different workload patterns. Hence, the workload prediction model should not tend to get overfitted to a specific workload pattern.
\item As the workload dataset grows with time, the predictive model should evolve and continuously learn the latest workload characteristics.
\item The workload predictor should be able to produce results within a bounded time.
\item Given that the time horizon for the prediction should be chosen based on the physical constraints like uptime and graceful shutdown time of Virtual Machines (VMs), the predictor should be able to produce sufficiently accurate results over a sufficiently large time horizon.
\end{itemize}

The objective of this work is to come up with a prediction method that can be trained in real-time to capture the latest trends and provide sufficiently accurate results for drastically different workload patterns. First, we evaluate the ability of existing models to produce accurate real-time predictions against evolving datasets. In this evaluation we tested time series forecasting methods, including statistical methods like ARIMA and exponential model, machine learning models like neural networks, and a prediction method used in Apache Stratos, an open source PaaS framework. Through this, we demonstrate the limitations in existing solutions in accurately predicting real-world workload traces, and highlight the need for more accurate predictions. Next, we propose an ensemble technique which combines results from a neural network, ARIMA and exponential models. Based on simulations with three publicly available workload traces (two real-world cloud workload datasets and the Google cluster dataset), we demonstrate that the proposed ensemble technique outperforms each of the tested individual methods and two other ensemble techniques as well.

The rest of the paper is structured as follows: Section 2 outlines related work on cloud workload prediction and their limitations. Section 3 describes and evaluates several existing time series forecasting techniques while exploring their ability to provide accurate predictions on publicly available datasets with predictable patterns under real-time training scenarios. Section 4 introduces the proposed ensemble technique and prediction algorithm. Section 5 presents the performance analysis, and concluding remarks are presented in Section 6.