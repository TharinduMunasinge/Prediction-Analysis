\section{Conclusion}

Providing a better QoS while maintaining lower resource lease costs in a PaaS system is extremely challenging for a reactive, threshold based auto scaling solution. Even though proactive solutions can overcome these challenges, they are dependent on highly accurate prediction mechanisms. Existing researches on workload prediction techniques mainly focus on single, offline-trained models. For heterogeneous  applications with drastically different and continuously growing workload statistics, forecasting method based on single models would not provide sufficiently accurate results, while they may also suffer from large errors for some workload patterns. When there is uncertainty in finding the best model, combining may reduce the instability of the forecast and therefore improve prediction accuracy.\\

In this work we have proposed an ensemble prediction technique that would combine predicted values from different prediction techniques based on the inverses of the past forecasting errors in each model. We train each individual model real time (as in a functional auto scaler) and combine the forecasts based on the weights calculated from the inverse errors of fitted values for the training data. Rather than taking only the last forecasting error or the mean over all past errors, we have used exponential smoothing to fit the past forecasting error of each model. Our observations have shown that using the inverse of the smoothed error as the contribution from each model significantly improves overall accuracy.  We have implemented the proposed ensemble technique using four popular forecasting models and tested it on three publicly available datasets, two cloud workload dataset and the Google cluster dataset.  We have demonstrated that our approach produces the least error in some datasets while never performing worse than any individual prediction methods in other cases.

Since the proposed forecasting method should generate the forecast within a bounded time period, we have limited the size of input training window. The optimum size of the input window that would maximize accuracy while not violating the temporal restrictions on calculating forecasts real-time needs to be researched upon as an improvement for this technique. It should also be pointed out that it is not desirable to blindly combine all available prediction models into a single ensemble model.