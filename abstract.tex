Elasticity is a key feature of cloud computing where allocation and deallocation of resources according to the user’s demand is done via auto scaling. Reactive auto scaling solutions, in which the scaling actions are taken place just after the triggering thresholds are met, suffer from several issues like risk of under provisioning at peak loads and resource wastage during non-peak loads. Proactive scaling solutions, where users’ future resource demands can be forecast and necessary scaling actions enacted beforehand, can fix these issues. Nevertheless, effectiveness of such proactive scaling solutions highly depends on the accuracy of the the prediction method used in the forecast.
In this paper we propose a workload forecasting technique which addresses the challenges specific to the domain of workload forecasting in a cloud auto scaler. We introduce an ensemble workload prediction mechanism based on time series and machine learning techniques for making accurate predictions on drastically different workload patterns. In this work, we initially evaluate several forecasting models for their applicability in forecasting different patterns. The proposed ensemble technique is implemented using three well-known forecasting models and tested for three real-world workload datasets. Forecasting accuracy is compared among the proposed solution, each of the individual models, and two other popular ensemble techniques. Experimental results have revealed that our ensemble method provides significantly lower forecast errors compared to other tested combinations.