Elasticity is a key feature of cloud computing where allocating and deallocating of resources according to the user’s demand is done via auto-scaling. The reactive autoscaling solutions in which the scaling actions are taken place just after the triggering thresholds are met, suffer from several issues like risk of under provisioning at peak load and wasting resources during non-peak load. Proactive scaling solutions in which users’ resource demand can be forecasted for future time and made necessary scaling actions beforehand can fix those issues. Nevertheless how effectively a proactive scaling solution can fix the issues reactive scaling solutions, is highly depends on the accuracy of the the prediction method used in proactive solution. In this paper we propose a workload forecasting technique which address the challenges specific to the workload forecasting in an cloud autoscaler. We introduce an ensemble workload prediction mechanism based on time series and machine learning techniques for making accurate predictions on drastically different workload patterns.  In this work, several forecasting models are evaluated for their applicability in forecasting drastically different patterns. The proposed ensembling technique was implemented using three well-known forecasting models and tested for three real-world workload datasets. Forecasting accuracy is compared among  the proposed solution, each individual models and two other popular ensemble techniques.  According to the experimental results, our ensemble scheme provides significantly lower forecast errors compared to the individual models and two other ensemble techniques.