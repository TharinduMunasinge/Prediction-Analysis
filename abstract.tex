Elasticity is a key feature of cloud computing where resources are allocated and released according to user demands. Reactive auto scaling, in which the scaling actions take place just after meeting the triggering thresholds, suffers from several issues like risk of under provisioning at peak loads and over provisioning during other times. Proactive scaling solutions, where a user's future resource demand can be forecast and necessary scaling actions enacted beforehand, can overcome these issues. Nevertheless, the effectiveness of such proactive scaling solutions depends on the accuracy of the prediction method(s) adopted. We propose a forecasting technique to enhance the accuracy of workload forecasting in cloud auto scalers. An ensemble workload prediction mechanism based on time series and machine learning techniques is proposed to make more accurate predictions on drastically different workload patterns. In this work, we initially evaluated several forecasting models for their applicability in forecasting different workload patterns. The proposed ensemble technique is then implemented using three well-known forecasting models and tested for three real-world workloads. Simulation results show that our ensemble method produces significantly lower forecast errors compared to the use of individual models and the prediction technique employed in Apache Stratos, an open source PaaS platform.

\textit{Keywords}--cloud computing; auto scaling; workload prediction; time series analysis