\usepackage{multirow}
\usepackage{booktabs}
\usepackage{multirow}
\usepackage[normalem]{ulem}

\section{Experimental Results}

We have implemented the prediction algorithm in R using the time series and machine learning model implementations in forecast package. We used the publicly available datasets mentioned in <<SECTION 1>> and also we have tested on several real world cloud workload datasets [GITHUB LINK FOR DATASET] collected from server applications.
% Please add the following required packages to your document preamble:

\begin{table}[]
\centering
\caption{My caption}
\label{my-label}
\begin{tabular}{|l|l|r|r|r|r|r|}
\hline
\multicolumn{1}{|c|}{\multirow{2}{*}{Model}} & \multicolumn{2}{c|}{sunspotarea}                        & \multicolumn{2}{c|}{Euretail}                               & \multicolumn{2}{c|}{oil}                                     \\ \cline{2-7} 
\multicolumn{1}{|c|}{}                       & \multicolumn{1}{c|}{MSE} & \multicolumn{1}{c|}{MAPE}    & \multicolumn{1}{c|}{MSE}     & \multicolumn{1}{c|}{MAPE}    & \multicolumn{1}{c|}{MSE}      & \multicolumn{1}{c|}{MAPE}    \\ \hline
Arima                                        & 382.36019                & 1.35863                      & 0.52439                      & 0.00417                      & 55.31262                      & 0.25081                      \\ \hline
Exponential                                  & 505.75034                & 1.16073                      & 0.57553                      & 0.01071                      & 54.98904                      & 0.25078                      \\ \hline
Nnet                                         & 473.92407                & 0.46524                      & 1.88208                      & 0.00618                      & 51.61619                      & 0.15951                      \\ \hline
Current                                      & 546.93791                & 0.96513                      & 0.65035                      & 0.00391                      & 61.80686                      & 0.58458                      \\ \hline
Ensamble                                     & 356.31452                & \multicolumn{1}{l|}{0.39285} & \multicolumn{1}{l|}{0.53734} & \multicolumn{1}{l|}{0.00309} & \multicolumn{1}{l|}{50.14736} & \multicolumn{1}{l|}{0.25598} \\ \hline
\end{tabular}
\end{table}

% Please add the following required packages to your document preamble:
% \usepackage{multirow}
\begin{table}[]
\centering
\caption{My caption}
\label{my-label}
\begin{tabular}{|l|r|r|r|l|l|l|}
\hline
\multicolumn{1}{|c|}{\multirow{2}{*}{Model}} & \multicolumn{2}{c|}{airmiles}                                   & \multicolumn{2}{l|}{Memory} & \multicolumn{2}{l|}{CPU} \\ \cline{2-7} 
\multicolumn{1}{|c|}{}                       & \multicolumn{1}{c|}{MSE}         & \multicolumn{1}{c|}{MAPE}    & MSE           & MAPE        & MSE         & MAPE       \\ \hline
Arima                                        & 1,412.94416                      & 0.72068                      & 7.59887       & 0.07539     & 2.97641     & 0.03593    \\ \hline
Exponential                                  & 1,370.47426                      & 0.43516                      & 8.86926       & 0.02381     & 3.15038     & 0.04807    \\ \hline
Nnet                                         & 2,400.57153                      & 0.71841                      & 7.13630       & 0.15322     & 2.79244     & 0.03130    \\ \hline
Current                                      & 1,367.01715                      & 0.92456                      & 11.73576      & 0.04637     & 5.69167     & 0.02372    \\ \hline
Ensamble                                     & \multicolumn{1}{l|}{1,361.41694} & \multicolumn{1}{l|}{0.58676} & 7.25582       & 0.09773     & 2.87274     & 0.02738    \\ \hline
\end{tabular}
\end{table}


Highlighted cells contains the smallest error value for each column(dataset). where as underlined cells contains the worst performance from all the methods. 

According to the results obtained, Stratos existing model has largest error in many datasets while each individual model has largest error atleaset in one dataset. Our proposed prediction technique provides the best prediction resuts in couple of datasets while it doesn't perform worser than individual prediction methods. 
